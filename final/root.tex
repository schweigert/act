\documentclass[letterpaper, 10 pt, conference]{ieeeconf}

\IEEEoverridecommandlockouts                              % This command is only needed if
   % you want to use the \thanks command

\overrideIEEEmargins                                      % Needed to meet printer requirements.

% See the \addtolength command later in the file to balance the column lengths
% on the last page of the document

% The following packages can be found on http:\\www.ctan.org
%\usepackage{graphics} % for pdf, bitmapped graphics files
%\usepackage{epsfig} % for postscript graphics files
%\usepackage{mathptmx} % assumes new font selection scheme installed
%\usepackage{times} % assumes new font selection scheme installed
%\usepackage{amsmath} % assumes amsmath package installed
%\usepackage{amssymb}  % assumes amsmath package installed
\usepackage[brazilian]{babel}
\usepackage[utf8]{inputenc}
\usepackage[T1]{fontenc}

\title{\LARGE \bf
Reconstrução de modelos 3D para uso na medicina
}


\author{Marlon Henry Schweigert \& Roberto Silvio Ubertino Rosso }


\begin{document}



\maketitle
\thispagestyle{empty}
\pagestyle{empty}


%%%%%%%%%%%%%%%%%%%%%%%%%%%%%%%%%%%%%%%%%%%%%%%%%%%%%%%%%%%%%%%%%%%%%%%%%%%%%%%%
\begin{abstract}
Este pseudo artigo visa buscar as últimas tecnologias utilizando reconstrução 3D para auxilio médico.
\end{abstract}


%%%%%%%%%%%%%%%%%%%%%%%%%%%%%%%%%%%%%%%%%%%%%%%%%%%%%%%%%%%%%%%%%%%%%%%%%%%%%%%%
\section{INTRODUÇÃO}


\section{PROCESSO DE RECONSTRUÇÃO}

A fim de reconstruir uma malha 3D, se faz necessário um conjunto de pontos que pertença ao objeto que será reconstruido.
%
Nesse sentido, se faz necessário um conjunto de imagens de tamanho para que supra a definição do algoritmo utilizado.

Este conjunto de imagens servem para gerar uma nuvem de pontos, a fim de ser processado por um algoritmo de reconstrução 3D.
%
Entretanto, este conjunto de imagens conterá uma quantidade de elementos massivos para garantir uma boa resolução do objeto.
%
Por este motivo, existe a necessidade de acelerar este processo por meio de algoritmos mais eficientes.

\bibliographystyle{IEEEtran}
\bibliography{IEEEabrv,IEEEexample}

\end{document}
